%!TEX root = main.tex
%
% simplicialcomplexes.tex
%

\chapter{Simplicial Sets}

In this preliminary section we touch upon some concepts regarding simplicial complexes. In some way they are just a special case of a presheaf topos. On the other hand, they have some intrinsic properties not found in other presheaf toposes. Most of this section is from \cite{goersjardinne09} and \cite{JT2008}. The aim of this section is to define the notion of a star, the nerve of a category and the geometric realization.

\section{Basic Definitions}

\begin{definition}
\label{def: finite cardinal}
Let $n \geq 0$ be a natural number. By the boldface $\mathbf{n}$ we mean the \emph{category} consisting of precisely $n+1$ objects, denoted by $0,1,2,\ldots,n-1,n$, and whose morphisms are precisely
\[ 0 \to 1 \to \cdots \to (n-1) \to n. \]
so that $\mathbf{n}$ may be regarded as a totally ordered set.
\end{definition}

\begin{definition}
\label{def:order category}
By $\mathbf{\Delta}$ we mean the category whose objects are the $\mathbf{n}$'s and whose morphisms are functors $\theta : \mathbf{n} \to \mathbf{m}$.
\end{definition}
If $\theta : \bf{n} \to \bf{m}$ is a morphism in $\bf{\Delta}$, we can also view it as an order-preserving function on the totally ordered sets $\bf{n}$ and $\bf{m}$. It is fruitful to confuse the two viewpoints sometimes.
Among the morphisms in $\mathbf{\Delta}$ are two special types, namely the coface and codegeneracy maps.
\begin{definition}
\index{coface map}
\index{codegeneracy map}
\label{def:coface and codegeneracy maps}
Let $\bf{n} \in \bf{\Delta}$ and let $j \in \bf{n}$. The $j$'th \emph{coface} map is the morphism
\[ d^j : \bf{n-1} \to \bf{n} \]
defined by
\[ d^j\left(0 \to \cdots \to (n-1) \right) = 0 \to \cdots \to (j-1) \to (j+1) \to \cdots \to (n+1), \]
i.e., ``skip $j$'', and the $j$'th \emph{codegeneracy} map is the morphism
\[ s^j : \bf{n+1} \to \bf{n} \]
defined by
\[ s^j\left(0 \to \cdots \to (n+1) \right) = 0 \to \cdots \to (j-1) \to j \to j \to (j+1) \to \cdots \to n, \]
i.e., ``repeat $j$''.
\end{definition}
If $\theta : \bf{n} \to \bf{m}$ is an order-preserving map, one can decompose it uniquely into a composite $\theta = \alpha \circ \beta$ where $\beta : \bf{n} \to \bf{p}$ is surjective and $\alpha : \bf{p} \to \bf{m}$ is injective. Then $\beta$ may be decomposed into a bunch of codegeneracy maps and $\alpha$ may be decomposed into a bunch of coface maps. So the morphisms in $\bf{\Delta}$ are generated by the coface and codegeneracy maps. Moreover, we have
\begin{lemma}[The Cosimplicial Identities]
\index{cosimplicial identities}
\label{lem:cosimplicial identities}
Let $n \geq 2$. Consider the diagram
\[ \begin{tikzcd}
\bf{n} \arrow[shift left=0.2em]{r}{s^j} \arrow[shift left=0.2em]{d}{s^i} & \bf{n-1} \arrow[shift left=0.2em]{l}{d^j} \arrow[shift left=0.2em]{d}{s^i} \\
\bf{n-1} \arrow[shift left=0.2em]{u}{d^i} \arrow[shift left=0.2em]{r}{s^{j-1}} & \bf{n-2} \arrow[shift left=0.2em]{u}{d^i} \arrow[shift left=0.2em]{l}{d^{j-1}}
\end{tikzcd} \]
in $\bf{\Delta}$. Then
\begin{align*}
d^j d^i &= d^i d^{j-1} & \forall \; i < j \\
s^{j-1} s^i &= s^i s^j & \forall \; i < j \\
s^j d^i &= d^i s^{j-1} & \forall \; i < j \\
s^i d^i &= \id & \; \\
s^i d^j &= d^{j-1} s^i & \forall \; i < j-1 \\
s^i d^{i+1} &= \id & i=j-1
\end{align*}
Moreover, the square of $s$'s is an absolute pushout, and the square of $d$'s is an absolute pullback.
\end{lemma}
\begin{proof}
The proof is easy. It's just a lot of bookkeeping.
\end{proof}
Recall that a (co)limit is termed absolute when it is preserved by \emph{any} functor. Since every morphism $\theta : \bf{n} \to \bf{m}$ factors uniquely into a composite of coface and codegeneracy maps, we conclude from the previous lemma
\begin{theorem}
\label{thm:absolute colimits and limits in the order category}
The category $\bf{\Delta}$ has absolute pushouts of surjections and absolute non-empty intersections of injections.
\end{theorem}

\begin{definition}
\index{standard $n$-simplex}
The category of \emph{simplicial sets} $\bf{S}$ is defined to be $\mathbf{Set}^{\bf{\Delta}^{op}}$.
\end{definition}
In practice, a simplicial set $X \in \bf{S}$ is determined when for each natural number $n$ a set $X_n$ is given together with face maps $d_i : X_{n+1} \to X_{n}$ and degeneracy maps $s_i : X_{n-1} \to X_{n}$.

\begin{definition}
The \emph{standard $n$-simplex} is the simplicial set $\Hom_{\Delta}(-,\bf{n}) = \bf{\Delta}(-,\bf{n}) = \bf{y}\bf{n}$.
\end{definition}

By a standard application of Yoneda's lemma we immediately obtain

\begin{lemma}
Let $X \in \bf{S}$ be a simplicial set. Then for each natural number $n \geq 0$ we have
\[ X_n \cong \Hom_{\mathbf{S}}\left( \mathbf{\Delta}(-,\mathbf{n}) , X \right). \]
\end{lemma}

\begin{definition}
\index{degenerate simplex}
\index{non-degenerate simplex}
An $n$-simplex $\sigma \in X_n$ is called \emph{degenerate} if there is a surjection $f : \mathbf{n} \to \mathbf{m}$ with $m < n$ and an $m$-simplex $\tau \in X_m$ such that $\sigma = X(f)(\tau)$. A simplex is called \emph{non-degenerate} if it is not degenerate.
\end{definition}

A very useful lemma which we shall use over and over again is the following.

\begin{lemma}[Eilenberg-Zilber]
\label{lem:eilenberg-zilber}
For each $n$-simplex $\sigma \in X_n$ there exists a unique order-preserving surjection $f : \mathbf{n} \to \mathbf{m}$ and a unique non-degenerate $m$-simplex $\tau \in X_m$ such that $\sigma = X(f)(\tau)$.
\end{lemma}
\begin{proof}
We follow \cite[Proposition 1.2.2]{JT2008}.
If $\sigma$ is already non-degenerate, take $m=n$ and $f = \id$. If $\sigma$ is degenerate, then there exists some surjection $f_1 : \mathbf{n} \to \mathbf{m}_1$ with $m_1 < n$ and an $m_1$-simplex $\sigma_1$ such that $\sigma = X(f_1)(\sigma_1)$. If $\sigma_1$ is non-degenerate, we are done. Otherwise, continue in this way until we reach a $\sigma_j$ that is non-degenerate. This process ends eventually since with each iteration, $m_i < m_{i-1}$, and vertices are always non-degenerate.
So the existence of such a pair $(f,\tau)$ with $\sigma = X(f)(\tau)$ is established. Suppose then that $(f',\tau')$ is another pair with $\sigma = X(f')(\tau')$. Consider the pushout
\[ \begin{tikzcd}
\mathbf{n} \arrow{r}{f} \arrow[swap]{d}{f'} & \mathbf{m} \arrow{d}{\pi} \\
\mathbf{m}' \arrow[swap]{r}{\pi'} & \mathbf{p}
\end{tikzcd} \]
in $\mathbf{\Delta}$. We can apply the Yoneda functor to get a commutative diagram
\[ \begin{tikzcd}
\mathbf{\Delta}\left(-,\mathbf{n}\right) \arrow{r}{\mathbf{\Delta}\left(-,f\right)} \arrow[swap]{d}{\mathbf{\Delta}\left(-,f'\right)} & \mathbf{\Delta}\left(-,\mathbf{m}\right) \arrow{d}{\mathbf{\Delta}\left(-,\pi\right)} \\
\mathbf{\Delta}\left(-,\mathbf{m}'\right) \arrow[swap]{r}{\mathbf{\Delta}\left(-,\pi'\right)} & \mathbf{\Delta}\left(-,\mathbf{p}\right)
\end{tikzcd} \]
in $\mathbf{S}$. This diagram is still a pushout by \cref{thm:absolute colimits and limits in the order category}. While the $n$-simplex $\sigma$ is an element of the set $X_n$, it may equivalently be regarded as a functor $\sigma : \mathbf{\Delta}(-,\mathbf{n}) \to S$ describing how the simplex is laid into the simplicial set. Similarly for $\tau$ and $\tau'$. Hence we have a commutative diagram
\[ \begin{tikzcd}
\mathbf{\Delta}\left(-,\mathbf{n}\right) \arrow{r}{\mathbf{\Delta}\left(-,f\right)} \arrow[swap]{d}{\mathbf{\Delta}\left(-,f'\right)} & \mathbf{\Delta}\left(-,\mathbf{m}\right) \arrow{d}{\mathbf{\Delta}\left(-,\pi\right)} \arrow[bend left]{ddr}{\tau} \\
\mathbf{\Delta}\left(-,\mathbf{m}'\right) \arrow[swap]{r}{\mathbf{\Delta}\left(-,\pi'\right)} \arrow[swap, bend right]{rrd}{\tau'} & \mathbf{\Delta}\left(-,\mathbf{p}\right) \arrow[dotted]{dr}{\exists ! \rho} \\
\; & \; & S
\end{tikzcd} \]
Since $\tau \circ \mathbf{\Delta}(-,f) = \sigma = \tau' \circ\mathbf{\Delta}(-,f')$, there exists a unique mediating $\rho$ such that $\mathbf{\Delta}(-,\pi) \circ \rho = \tau$ and $\mathbf{\Delta}(-,\pi') \circ \rho = \tau'$ as indicated by the dotted arrow. 
We can regard $\rho$ equivalently as a $p$-simplex in the set $X_p$, while $\pi$ and $\pi'$ are order-preserving surjections. So we see that $\tau = X(\pi)(\rho)$ and $\tau' = X(\pi')(\rho)$. But $\tau$ and $\tau'$ are both non-degenerate. So we must have $\pi = \pi' = \id$, $m=p=m'$, and hence $\tau = \tau'$.
\end{proof}

Recall that every presheaf is a colimit of representable presheaves by \cref{prop:every presheaf is a colimit of representables}. The good news is that there are not many different representable presheaves in $\bf{S}$. So we conclude

\begin{lemma}
\label{lem:every simplicial set is a colimit of standard n simplices}
If $X \in \bf{S}$ is a simplicial set, there is a canonical isomorphism
\[ X \cong \colim \left( \int_{\mathbf{\Delta}} X \xrightarrow{\pi_X} \mathbf{\Delta} \xrightarrow{\mathbf{y}} \mathbf{Set}^{\mathbf{\Delta}^{op}} \right) \]
\end{lemma}
The above lemma tells you ``how the standard simplices comprise $X$'', and is very combinatorial in nature. It's often convenient to view \cref{lem:every simplicial set is a colimit of standard n simplices} as a different (but equivalent, of course) colimit. Namely, define $\mathbf{\Delta}  \downarrow  X$ to be the category whose objects are maps $\sigma : \mathbf{\Delta}(-,\mathbf{n}) \to X$ (or just simplices of $X$). An arrow in $\mathbf{\Delta}  \downarrow  X$ is a commutative diagram of simplicial maps
\[ \begin{tikzcd}
\mathbf{\Delta}(-,\mathbf{n}) \arrow[swap]{dd}{\theta} \arrow{dr}{\sigma} & \; \\
 & X \\
\mathbf{\Delta}(-,\mathbf{m}) \arrow[swap]{ur}{\tau} & \;
\end{tikzcd} \]
It is not hard to prove that the colimit in \cref{lem:every simplicial set is a colimit of standard n simplices} is the same thing as
\begin{lemma}
\[ X \cong \colim_{\stackrel{\mathbf{\Delta}(-,\mathbf{n}) \to X}{\text{in } \mathbf{\Delta}\downarrow X}} \mathbf{\Delta}(-,\mathbf{n}). \]
\end{lemma}

\section{Geometric Realization}

\begin{definition}
\index{standard geometric $n$-simplex}
\label{def:standard geometric n-simplex}
The \emph{standard geometric $n$-simplex} is the topological space
\[ \Delta^n := \left\{ (t_0,\ldots,t_n) \in \R^{n+1} \mid \sum_{i=0}^n t_i = 1,\; t_i \geq 0 \text{ for all } i \right\} \]
endowed with the subspace topology from the metric space $\R^{n+1}$.
\end{definition}
There is a functor $|-| : \mathbf{\Delta} \to \mathbf{Top}$ defined by sending $\bf{n}$ to the standard geometric $n$-simplex in $\R^{n+1}$. It's easy to visualize what $|-|$ should do with the coface and codegeneracy maps.
As with all such constructions, once we have a way to turn representable objects into an object of interest in a category, we have a way to turn every presheaf into an object of interest in a category. For a more precise meaning of the previous sentence, we have the following definition.
\begin{definition}
\index{geometric realization}
\label{def:geometric realization of a simplicial set}
The \emph{geometric realization} of $X$ is defined to be the colimit of spaces
\[ |X| := \colim_{\stackrel{\mathbf{\Delta}(-,\mathbf{n}) \to X}{\text{in } \mathbf{\Delta}\downarrow X}} \Delta^n \in \mathbf{Top} \]
\end{definition}
Elements of $|X|$ are equivalence classes of pairs $[\sigma, t]$, where $\sigma \in X_n$ is an $n$-simplex and $t \in \Delta^n$. The equivalence relation is defined by the rule
\[ \left(X(\theta)(\sigma), t \right) \sim \left( \sigma, |\theta|(t) \right), \qquad \theta : \mathbf{n} \to \mathbf{m}. \]
In fact, this is reminiscent of the tensor product! Unfortunately, $\mathbf{Top}$ is not a topos. Moreover, $|X|$ is a space, not just a set (as it is for the tensor product). So we cannot formally define the geometric realization in this way.
A map $f : X \to Y$ of simplicial sets (so, a natural transformation between the presheaves) determines a functor
\[ \mathbf{\Delta} \downarrow f : \mathbf{\Delta} \downarrow X \to \mathbf{\Delta} \downarrow Y, \qquad \sigma \mapsto f \circ \sigma. \]
So we get a continuous map
\[ |f| : |X| \to |Y|, \qquad [\sigma,t] \mapsto [f \circ \sigma, t]. \]
Hence $|-| : \mathbf{\Delta} \to \mathbf{Top}$ extends to a functor $|-| : \mathbf{S} \to \mathbf{Top}$. There's also a functor going the other way. Observe that for any given topological space $S \in \mathbf{Top}$ we have a simplicial set $\Sing(S)$ defined on objects by $\Sing(S)_n := \Hom_{\mathbf{Top}}\left(\Delta^n, S\right)$. It's an exercise to determine what $\Sing$ should do with continuous maps, and what the face and degeneracy maps are for $\Sing(S)$.

\begin{theorem}
The functor $|-|$ is the left-adjoint of $\Sing$.
\end{theorem}
\begin{proof}
There are natural bijections
\begin{align*}
\Hom_{\mathbf{Top}}\left(|X|, S \right) &= \Hom_{\mathbf{Top}}\left( \colim_{\stackrel{\mathbf{\Delta}(-,\mathbf{n}) \to X}{\text{in } \mathbf{\Delta}\downarrow X}} \Delta^n, S \right) \\
&\cong \lim_{\stackrel{\mathbf{\Delta}(-,\mathbf{n}) \to X}{\text{in } \mathbf{\Delta}\downarrow X}} \Hom_{\mathbf{Top}}\left( \Delta^n, S \right) \\
&\cong \lim_{\stackrel{\mathbf{\Delta}(-,\mathbf{n}) \to X}{\text{in } \mathbf{\Delta}\downarrow X}} \Hom_{\mathbf{S}}\left( \mathbf{\Delta}(-,\mathbf{n}), \Sing(S) \right) \\
&\cong \Hom_{\mathbf{S}}\left(X, \Sing(S) \right).
\end{align*}
\end{proof}

\section{The Nerve of a Category}
From the way we defined the category $\mathbf{\Delta}$ we have an obvious inclusion functor
\[ \mathbf{\Delta} \to \mathbf{Cat}, \qquad \mathbf{n} \mapsto \mathbf{n}. \]
So we have a way to change representable objects into objects of a category that we are interested in. We can do the whole construction again, but this time in $\mathbf{Cat}$ instead of $\mathbf{Top}$. It's possible to set up an adjoint pair of functors between $\mathbf{S}$ and $\mathbf{Cat}$ just like for the adjoint pair $|-| \dashv \Sing$. We will not be needing the left adjoint. What we will be needing from this adjunction, though, is the right adjoint. We define formally
\begin{definition}
\index{nerve}
\label{def:nerve of a category}
Let $\mathbf{C} \in \mathbf{Cat}$ be a small category. We define the \emph{nerve} of $\mathbf{C}$ to be the simplicial set $N\mathbf{C}$ defined for each natural number $n \geq 0$ as the set
\[ (N\mathbf{C})_n := \Hom_{\mathbf{Cat}}\left(\mathbf{n}, \mathbf{C} \right). \]
\end{definition}
Spelled out in detail, this means that the $0$-simplices are the objects of $\mathbf{C}$ (which is indeed a set). If $n>0$, then an $n$-simplex of $(N\mathbf{C})_n$ is an $n$-tuple
\[ A_0 \xrightarrow{f_1} A_1 \xrightarrow{f_2} \cdots \xrightarrow{f_n} A_n \]
of composable morphisms $f_i : A_{i-1} \to A_i$. We'll denote such simplices more compactly by $(f_1,\ldots,f_n)$. The face maps
\[ d_j : (N\mathbf{C})_{n} \to (N\mathbf{C})_{n-1} \] compose two consecutive morphisms at the $j$'th position to get an $(n-1)$-tuple of composable morphisms when $0 < j < n$, and they discard the outer morphism when $j = 0$ or $n$. The degeneracy maps
\[ s_j : (N\mathbf{C})_n \to (N \mathbf{C})_{n+1} \]
insert an identity morphism at the $j$'th position to get a degenerate $(n+1)$-tuple.

\section{Star Neighborhoods}

From this point on, the thesis consists of mostly original work, so the proofs become more detailed.

\begin{lemma}
\label{lem:subcomplex is closed subset in the realization}
Let $S \in \mathbf{S}$ be a simplicial complex and $T \subset S$ a subcomplex. Then $|T| \subset |S|$ is a closed set.
\end{lemma}
\begin{proof}
Let $\sigma \in S_n$ be an $n$-simplex of $S$. Let $I$ be the set of all possible faces of $\sigma$ which are contained in $T$. Note that $I$ is a finite set. Therefore $\bigcup_{\tau \in I}|\tau|$ is closed in $|\sigma|$.
\end{proof}

\begin{definition}
\index{star}
\label{def:star of a simplex}
Let $S \in \mathbf{S}$ be a simplicial set, and an $n$-simplex $\sigma \in S_n$. The \emph{star} of $\sigma$, denoted $\st(\sigma)$, is the set of all simplices in $S$ which contain $\sigma$ as an eventual face.
\end{definition}

This means that $\sigma \in S_n$ is an eventual face of some $\tau \in S_m$ with $m>n$ if and only if there exist natural numbers $i_1,i_2,\ldots,i_{m-n}$ such that $\sigma = (d_{i_{m-n}} \circ \cdots \circ d_{i_{2}} \circ d_{i_{1}})(\tau)$. If $m=n$, then $\sigma$ is a face of $\sigma$, so $\sigma \in \st(\sigma)$. If $m < n$, then $S_m \cap \st(\sigma) = \emptyset$. Put in a more functorial way, identify $\sigma$ and $\tau$ as functors $\sigma : \mathbf{\Delta}(-,\mathbf{n}) \to S$ and $\tau : \mathbf{\Delta}(-,\mathbf{m}) \to S$. Then $\sigma$ is an eventual face of $\tau$ if and only if there exists an injective order-preserving map $\theta : \mathbf{n} \to \mathbf{m}$ such that
\begin{equation}
\label{eq:sigma is eventual face of tau diagram}
\begin{tikzcd}
\mathbf{\Delta}(-,\mathbf{n}) \arrow{dr}{\sigma} \arrow[swap]{dd}{\mathbf{\Delta}(-,\theta)} & \\
& S \\
\mathbf{\Delta}(-,\mathbf{m}) \arrow[swap]{ur}{\tau}
\end{tikzcd}
\end{equation}
is a commutative diagram of simplicial sets.

\begin{definition}
\label{def:simplicial set minus a simplex}
Let $S \in \mathbf{S}$ and take an $n$-simplex $\sigma \in S_n$. The simplicial set $S - \sigma$ is defined as follows. For any $n \in \N$, we set
\[ (S-\sigma)_n := \left\{ \tau \in S_n : \tau \not\in \st(\sigma) \right\}. \]
The face and degeneracy maps for $S-\sigma$ are the same as from $S$, and by definition of the star they are well-defined.
\end{definition}

It is obvious that $S - \sigma \subset S$. As a consequence, $|S-\sigma|$ is a closed subspace inside $|S|$ by \cref{lem:subcomplex is closed subset in the realization}.

\begin{definition}
\index{realization of the star}
\label{def:realization of the star}
Let $S \in \mathbf{S}$ be a simplicial set and $\sigma \in S_n$ an $n$-simplex. We define the \emph{realization of the star} $\sigma^* \subset |S|$ as the open subspace $|S| - |S-\sigma|$.
\end{definition}
Observe that if $v \in S_0$ is a $0$-simplex (a vertex), then the only point $p \in v^*$ with the property that $p = |w|$ for some $w \in S_0$ is $v$.

% \begin{definition}
% Let $S \in \mathbf{S}$ be a simplicial set and take a non-degenerate $n$-simplex $\sigma \in S_n$. Identify $|\sigma|$ with its continuous map $\sigma : \Delta^n \to |S|$. Because $\sigma$ is non-degenerate, the map $|\sigma|$ is injective. We define the \emph{interior} of $\sigma$ as the image of $\{(t_0,\ldots,t_n) \in \Delta^n : t_i > 0\}$, denoted by $\Int(\sigma)$. If $\sigma$ is degenerate, there exists a unique non-degenerate $m$-simplex $\tau$ and a surjective order-preserving map $f : [n] \to [m]$ such that $f^* \tau = \sigma$. In particular, $|\tau|$ and $|\sigma|$ give the same realization in $|S|$. We define the \emph{interior} of the degenerate $n$-simplex $\sigma$ as the interior of the non-degenerate $m$-simplex $\tau$.
% \end{definition}

\begin{definition}
\index{interior}
\label{def:interior of a simplex}
Let $S \in \mathbf{S}$ be a simplicial set and take an $n$-simplex $\sigma \in S_n$. Identify the realization of $\sigma$ in $|S|$ with its continuous map $|\sigma| : \Delta^n \to |S|$. We define the \emph{interior} of $\sigma$ as the image of $\{(t_0,\ldots,t_n) \in \Delta^n : t_i > 0,\; i=0,\ldots,n\}$.
\end{definition}

Observe that this definition works fine even for degenerate simplices, and note also that the interior of a $0$-simplex is a point in $|S|$. Note also that the interior of a simplex is not necessarily open in $|S|$. However, we do have

\begin{lemma}
\label{lem:realization of the star is union of the interiors}
Let $\sigma \in S_n$ be an $n$-simplex of a simplicial set $S \in \mathbf{S}$. Then $\sigma^* = \bigcup_{\tau \in \st(\sigma)} \Int(\tau)$.
\end{lemma}
\begin{proof}
The inclusion $\bigcup_{\tau \in \st(\sigma)} \Int(\tau) \subseteq \sigma^*$ is trivial. Let us prove the other inclusion.

Let $p \in \sigma^*$. Then $p \not\in |S-\sigma|$. This means that $p$ is not in the (closed) image of any $m$-simplex $|\tau| : \Delta^m \to |S|$ for which $\sigma$ is \emph{not} an eventual face. 
But $p \in |S|$, so what remains is that $p$ is in the image of an $m$-simplex $|\tau| : \Delta^m \to |S|$ for which $\sigma$ is an eventual face. 
By \cref{lem:eilenberg-zilber}, we may assume $\tau$ to be non-degenerate. 
Let $(t_0,\ldots,t_m) \in \Delta^m$ be the barycentric coordinates such that $|\tau|(t_0,\ldots,t_m) = p$. 
If $m=0$ then we are done, so assume $m>0$. If $t_i = 0$ for some $i$, we may replace $\tau$ by its face $d_i \tau$ and replace the barycentric coordinates by $(t_0,\ldots,t_{i-1},t_{i+1},\ldots,t_m)$. After such a replacement, we still have $d_i \tau \in \st(\sigma)$. 
Continue in this way until we reach a $k$-simplex $\tau'$, with $k \leq m$, with barycentric coordinates $(t_0,\ldots,t_k)$ with $t_i > 0$ for all $i=0,\ldots,k$. 

\end{proof}

\begin{lemma}
\label{lem:realization of the star is contained in the intersection of the realizations of its faces}
Let $n>0$ and take an $n$-simplex $\sigma \in S_n$ in a simplicial set $S \in \mathbf{S}$. Then
\[ \sigma^* \subset \bigcap_{i=0}^n (d_i \sigma)^*. \]
\end{lemma}
\begin{proof}
If $\tau \in \st(\sigma)$, then $\tau$ is eventually a face of $\sigma$, so it's clearly also eventually a face of $d_i \sigma$ for $i=0,\ldots,n$. Hence $\tau \in \bigcap_{i=0}^n \st(d_i \sigma)$. Therefore $\Int(\tau) \subset (d_i \sigma)^*$ for every $i=0,\ldots,n$.
\end{proof}

\begin{lemma}
\label{lem:realization of the star is connected}
Let $S \in \mathbf{S}$. For every $n$-simplex $\sigma \in S_n$, the realization of the star $\sigma^* \subset |S|$ is a connected subset.
\end{lemma}
\begin{proof}
Suppose that we can write $\sigma^* = U \cup V$, where $U$ and $V$ are two disjoint non-empty open subsets of $\sigma^*$. Let $\tau \in \st(\sigma)$. Then $|\tau|$ is a compact connected metric space, so $\Int(\tau)$ is a connected metric space. but $\Int(\tau) = \left(U \cap \Int(\tau) \right) \cup \left(V \cap \Int(\tau) \right)$, so either $\Int(\tau) \subset U$ or $\Int(\tau) \subset V$. Let $I$ be the set of all $\tau \in \st(\sigma)$ for which $\Int(\tau) \subset U$ and let $J$ be the set of all $\tau \in \st(\sigma)$ for which $\Int(\tau) \subset V$. Both $I$ and $J$ are non-empty. By \cref{lem:realization of the star is union of the interiors}, we find a decomposition
\[ \sigma^* = \bigcup_{\tau \in \st(\sigma)} \Int(\tau) = \bigcup_{\tau \in I} \Int(\tau) \cup \bigcup_{\tau \in J} \Int(\tau). \]
Without loss of generality, $\sigma \in I$. If $\tau \in \st(\sigma)$, with $\tau$ an $m$-simplex, $n \leq m$, then there exists a commutative diagram as in \cref{eq:sigma is eventual face of tau diagram}. Thus, we have a commutative diagram of topological spaces and continuous maps
\[ \begin{tikzcd}
\Delta^n \arrow{dr}{|\sigma|} \arrow[swap]{dd}{|\mathbf{\Delta}(-,\theta)|} & \\
& \left| S \right| \\
\Delta^m \arrow[swap]{ur}{|\tau|}
\end{tikzcd} \]
By connectedness of $\Delta^m$ and $\Delta^n$, we have $\tau \in I$. But this holds for arbitrary $\tau$, so $J$ is empty. A contradiction.
\end{proof}

\begin{lemma}
\label{lem:realizations of stars form an open cover}
The realizations of the stars of vertices form an open cover of $|S|$. That is, $|S| = \bigcup_{v \in S_0} v^*$.
\end{lemma}
\begin{proof}
By \cref{lem:realization of the star is union of the interiors},
\[ \bigcup_{v \in S_0}v^* = \bigcup_{v \in S_0} \bigcup_{\tau \in \st(v)} \Int(\tau). \]
The right-hand-side of this equation exhausts all possible simplices of $S$. So it suffices to prove that every point $p \in |S|$ is contained in the interior of some $n$-simplex $\sigma$.
Let $p \in |S|$. Then $p \in |\sigma|$ for some $n$-simplex $\sigma \in S_n$. By \cref{lem:eilenberg-zilber}, there exists a unique non-degenerate $m$-simplex $\tau \in S_m$ and a surjective order-preserving map $f : \mathbf{n} \to \mathbf{m}$, with $m \leq n$, such that $(Sf)(\tau) = \sigma$. Thus $p \in |\tau|$. If $p \in \Int(\tau)$ then we are done. Otherwise, $p \in |d_i \tau|$ for some $i$. Continue in this way. Eventually this process stops.
\end{proof}

