%!TEX root = main.tex
%
% config.tex
%

\documentclass[11pt,a4paper,titlepage]{report}

\usepackage[pdfpagelabels]{hyperref}
\usepackage[utf8]{inputenc}

\usepackage
{
	makeidx,
	url,
	amsmath, 
	amssymb, 
	amsthm,
	mathrsfs,
	tikz,
	tikz-cd,
	chngcntr,
	apptools,
	cleveref,
	enumitem,
	cite,
	algorithm,
	algpseudocode,
	lmodern
}

% \usepackage[nottoc]{tocbibind}
\usepackage{etoolbox}
\makeatletter
\patchcmd{\l@chapter}{1.0em}{0.8em}{}{}
\makeatother

\usepackage{tikz-3dplot/tikz-3dplot}
\usepackage{pgfplots}
\usetikzlibrary{shapes}
\tdplotsetmaincoords{60}{0}
\usetikzlibrary{arrows}
\usetikzlibrary{shapes.geometric}

% \pgfplotsset{compat=1.13}
\pgfplotsset{compat=1.12}

\author{Raoul Wols}
\title{Master's Thesis}
\date{\today}

\theoremstyle{plain}
	\newtheorem{theorem}{Theorem}
	\counterwithin{theorem}{section} % important bit
	\newtheorem{lemma}[theorem]{Lemma}
	\newtheorem{proposition}[theorem]{Proposition}
	\newtheorem{corollary}[theorem]{Corollary}
	\newtheorem{claim}[theorem]{Claim}
	\newtheorem{conjecture}[theorem]{Conjecture}

	\newtheorem*{theorem*}{Theorem}
	\newtheorem*{lemma*}{Lemma}
	\newtheorem*{proposition*}{Proposition}
	\newtheorem*{corollary*}{Corollary}
	\newtheorem*{claim*}{Claim}
	\newtheorem*{conjecture*}{Conjecture}

\theoremstyle{definition}
	\newtheorem{definition}[theorem]{Definition}
	\newtheorem{example}[theorem]{Example}
	\newtheorem{exercise}{Exercise}
	\newtheorem{fact}[theorem]{Fact}
	\newtheorem{question}[theorem]{Question}
	\newtheorem{construction}[theorem]{Construction}

	\newtheorem*{definition*}{Definition}
	\newtheorem*{example*}{Example}
	\newtheorem*{exercise*}{Exercise}
	\newtheorem*{fact*}{Fact}
	\newtheorem*{question*}{Question}
	\newtheorem*{construction*}{Construction}

\theoremstyle{remark}
\newtheorem{remark}[theorem]{Remark}
\newtheorem{note}[theorem]{Note}

\AtAppendix{\counterwithin{theorem}{section}}
\AtAppendix{\counterwithin{exercise}{section}}

\newcommand{\N}{\mathbb{N}}
\newcommand{\Z}{\mathbb{Z}}
\newcommand{\Q}{\mathbb{Q}}
\newcommand{\R}{\mathbb{R}}
\newcommand{\C}{\mathbb{C}}
\newcommand{\NEEDREF}{\begin{color}{red}{\texttt{REF?}}\end{color}}
\newcommand{\TODO}{\begin{color}{red}{\texttt{TODO}}\end{color}}

\newcommand{\axiomnumbering}[1] {($#1 \arabic* $)}
\newcommand{\romansmallnumbering}{(\roman*)}


\DeclareMathOperator{\pr}{pr}
\DeclareMathOperator{\im}{im}
\DeclareMathOperator{\lcf}{lcf}
\DeclareMathOperator{\Int}{int}
\DeclareMathOperator{\ST}{\mu}
\DeclareMathOperator{\st}{star}
\DeclareMathOperator{\Sym}{Sym}
\DeclareMathOperator{\Stab}{Stab}
\DeclareMathOperator{\id}{id}
\DeclareMathOperator{\Hom}{Hom}
\DeclareMathOperator{\Iso}{Iso}
\DeclareMathOperator{\Aut}{Aut}
\DeclareMathOperator{\Sing}{Sing}
\DeclareMathOperator{\lcm}{lcm}
\DeclareMathOperator{\Spec}{Spec}
\DeclareMathOperator{\Sh}{Sh}
\DeclareMathOperator{\verteq}{\rotatebox{90}{=}}
\DeclareMathOperator{\op}{op}
\DeclareMathOperator{\Sub}{Sub}
\DeclareMathOperator*{\colim}{colim}
\DeclareMathOperator{\cod}{cod}
\DeclareMathOperator{\dom}{dom}
\DeclareMathOperator{\eval}{eval}
\DeclareMathOperator{\tr}{tr}
\DeclareMathOperator{\supp}{supp}

\newcommand{\HomLHOverXC}[2]{\Hom_{\mathbf{LH}/X_{\mathbf{C}}}\left(#1, #2\right)}
\newcommand{\uHomLHOverXC}[2]{\underline{\Hom}_{\mathbf{LH}/X_{\mathbf{C}}}\left(#1, #2\right)}
