%!TEX root = main.tex
%
% outlook.tex
%

\chapter{Outlook and Open Problems}

We have partially answered \cref{eq:intro-question} with \cref{thm:mccord functor generalizes the mccord map} and \cref{thm:intro-theorem}. Here we shall consider some problems that one might still want to solve.

\bigskip

First, there is the evidence from \cite{lenz2011} that \cref{thm:intro-theorem} also holds for monoid categories $\mathbf{C}$.
In order to move forward in that direction, perhaps it is best if \cref{conj:if C is a monoid then ST is flat} is answered positively.

\bigskip

The definition of an Alexandroff category (\cref{def:alexandroff cat}) is rather technical, and it would be benificial to provide a theorem that gives sufficient conditions to detect an Alexandroff category.

\bigskip

During the development of the theorems, the suspicion arose that there might be some redundancy in \cref{def:star sieve}. It would be a good approach to reconsider it and optimize away any redundant conditions.

\bigskip

The appendix in \cite{AlgebraicTopologyBible} treats a systematic way to build a space given some opens from it and equivalence relations on them. Perhaps the construction of the McCord space can be more streamlined by using that.

\bigskip

Finally, I feel that there is some inherent degeneracy in \cref{def:star sieve}. More precisely, we seem to only care about $1$-morphisms and thus $1$-simplices. The higher-order simplices are only there because the nerve functor creates them. Perhaps if one is willing to work with higher categories, more needs to be taken care of in \cref{def:star sieve}. For instance, one could imagine that there be not only a $1$-morphism, but also a $2$-morphism such that \ldots, a $3$-morphism such that \ldots, and so forth. The author hopes that in this way, higher homotopy groups of toposes will reveal themselves naturally.