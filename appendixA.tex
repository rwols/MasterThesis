%!TEX root = main.tex
%
% appendixA.tex
%

\chapter{Finite Topological Spaces}
Most can be found in \cite{barmak11}.
%-----------------------------------------------------------------------------
\begin{definition}
	Denote the category of finite topological spaces by $\mathbf{FinTop}$.
	The morphisms are continuous maps.
	Denote the category of finite preordered sets by $\mathbf{FinPre}$.
	The morphisms are order-preserving maps. That is, $f : X \to Y$ is a
	morphism between preordered sets if and only if
	$x \leq x' \Longrightarrow f(x) \leq f(x')$.
	Denote the category of finite directed graphs by $\mathbf{FinGraph}$.
\end{definition}
%-----------------------------------------------------------------------------
\begin{definition}
	If $X \in \mathbf{FinTop}$ and $x \in X$, define the \emph{open hull} of
	$x$ as
	\[ U_x := \bigcap_{x \in U \subseteq X} U,\]
	where $U$ ranges over all open neighborhoods of $x$. 
	If necessary, we may write $U_x^X$ to make clear that we are talking about
	the open hull of a specific space $X$. 
	Similarly, we define the \emph{closed hull} of $x$ as
	\[ F_x := \bigcap_{x \in F \subseteq X} F,\]
	where $F$ ranges over all closed neighborhoods of $x$.
\end{definition}
%-----------------------------------------------------------------------------
\begin{remark}
	The open hull $U_x$ of a point $x \in X$ is an open set. 
	It is the smallest open set that contains the point $x$. 
	Similarly, $F_x$ is the smallest closed set that contains $x$.
	Observe too, that $F_x = \overline{\{x\}}$.
\end{remark}
%-----------------------------------------------------------------------------
\begin{lemma} \label{lem:closed hulls and open hulls}
	Let $X \in \mathbf{FinTop}$ and $x,y \in X$.
	Then $x \in F_y \iff y \in U_x$.
\end{lemma}
\begin{proof}
For two points $x,y \in X$, we have
\begin{align*}
x \in F_y &\iff x \in \overline{\{y\}} \\
&\iff x = y \text{ or } x \text{ is a limit point of } y \\
&\iff x = y \text{ or for every open } U \subseteq X \text{ with } 
	x \in U \text{ it also holds } y \in U \\
&\iff y \text{ is in every open neighborhood of } x \\
&\iff y \in U_x
\end{align*}
\end{proof}
Recall that a topological space $X$ is called a $T_0$-space if for every two
points $x,y \in X$, with $x \neq y$ there exists an open set $U \subseteq X$
such that $x \in U$ and $y \not\in U$. If $X$ has the trivial topology, then
$X$ is not $T_0$. Every discrete space has the $T_0$-property. Every
Hausdorff space is $T_0$. Not every $T_0$-space is a Hausdorff space.
For instance, a set with two points $\{0,1\}$ where the topology is given
by
\[ \emptyset, \{0,1\}, \{1\} \]
is a $T_0$-space, but is not Hausdorff.
A $T_1$-space is a topological space $X$ such that for every two points
$x,y \in X$, with $x \neq y$ there exist two open sets $U,V \subseteq X$ such
that $x \in U$, $y \not\in U$ and $x \not\in V$, $y \in V$. Clearly a space
being $T_1$ is also $T_0$. Not every $T_1$-space is Hausdorff. For instance,
take $\Z$ with the cofinite topology. More generally, every irreducible
$T_1$-space with more than one point is not Hausdorff.
%-----------------------------------------------------------------------------
\begin{definition}
	Given $X \in \mathbf{FinTop}$, we define a relation
	\[ x \leq y \iff x \in F_y. \]
\end{definition}
%-----------------------------------------------------------------------------

Given $P \in \mathbf{FinPre}$, we can also construct a graph 
$(P,E) \in \mathbf{FinGraph}$ where the vertices are the elements of $P$
and we construct an edge $(x,y) \in E$ if $x \geq y$. A graph morphism
between two directed graphs $(V,E),(V',E') \in \mathbf{FinGraph}$ is a map
of sets $f : V \to V'$ with the property that whenever $(v,w) \in E$, then
$(f(v),f(w)) \in E'$. In other words, graph morphisms respect edges.

Given $(V,E) \in \mathbf{FinGraph}$, we can also decree the sets
\[ U_x := \{ y \in V : y \leq x \}, \qquad x \in V \]
as a basis of open sets for $V$, endowing $V$ with a topology.

These three constructions give us equivalent objects in their respective
categories.

%-----------------------------------------------------------------------------
\begin{proposition}
	Let $X \in \mathbf{FinTop}$.
	\begin{enumerate}[label=\romansmallnumbering]
		\item \label{prop:p1i} $x \leq y \iff F_x \subseteq F_y$.
		\item \label{prop:p1ii} $\leq$ is a preordering.
		\item \label{prop:p1iii} $X$ is $T_0 \iff \leq$ is a partial order 
			$\iff X$ has no directed cycles in its graph.
		\item \label{prop:p1iv} $X$ is $T_1 \iff X$ is discrete $\iff$ no
		two distinct points are comparable $\iff$ the graph of $X$ contains
		no edges.
	\end{enumerate}
\end{proposition}
\begin{proof}
For (\ref{prop:p1i}), suppose that $x \leq y$. Then $x \in F_y$. So $F_y$ is a
closed neighborhood for $x$. $F_x$ is the smallest closed neighborhood for
$x$. Hence $F_x \subseteq F_y$. On the other hand, suppose that
$F_x \subseteq F_y$. Then $x \in F_y$, so $x \leq y$. For (\ref{prop:p1ii}),
we have $x \in F_x$, so $x \leq x$. Suppose that $x \leq y$ and $y \leq z$ for
three points $x,y,z \in X$. Then by (\ref{prop:p1i}), we have
$F_x \subseteq F_y \subseteq F_z$, so $x \leq y$. For (\ref{prop:p1iii}),
Suppose that $X$ is $T_0$. 
Assume we have two points $x,y \in X$ such that
$x \leq y$ and $y \leq x$. 
Then $U_x = U_y$ by (\ref{prop:p1i}) and Lemma
\ref{lem:closed hulls and open hulls}. 
Since $U_x$ and $U_y$ are the smallest
open sets that contain $x,y$, there exists no open $U \subseteq X$ such that
$x \in U$ and $y \not\in U$. 
Therefore $x=y$. Hence $\leq$ is antisymmetric.
On the other hand, if $\leq$ is antisymmetric, assume we have two distinct
points $x,y \in X$.
If $x$ and $y$ are incomparable, we may take $U = X \setminus F_y$.
If $x$ and $y$ are comparable, let's say $x \leq y$ without loss of
generality.
Since $\leq$ is assumed to be a partial order and $x \neq y$, we know that
$x < y$, hence $y \not\in F_x \iff x \not\in U_y$ by Lemma
\ref{lem:closed hulls and open hulls}.
So we may take $U = U_y$.
\end{proof}
%-----------------------------------------------------------------------------
\begin{definition}
	Given $X \in \mathbf{FinPre}$, define for every $x \in X$ the set
	\[ U_x := \{y \in X : x \leq y \}. \]
\end{definition}
%-----------------------------------------------------------------------------
\begin{proposition}
Let $X \in \mathbf{FinPre}$. Then the set $\{U_x : x \in X\}$ is a basis of
open sets for a topology on $X$.
\end{proposition}
\begin{proof}
Clearly $\bigcup_{x \in X} U_x = X$. 
Let $x,y \in X$ be such that $U_x \cap U_y \neq \emptyset$.
If $z \in U_x \cap U_y$, then $x \leq z$ and $y \leq z$. So if $w \in U_z$,
then $w \leq z \leq x,y$, therefore $U_z \subseteq U_x \cap U_y$.
\end{proof}
To every finite space $X$, we can associate a simplicial set. For any $n \in \N$, define
\[ [n] := \left\{0,1,2,\ldots,n-1,n\right\} \subset \N. \]
The subsets of the form $[n]$ together with monotonic maps between them forms a category denoted by $\Delta$. In $\Delta$, define the morphisms
\[ d^i_n : [n-1] \to [n], \qquad j \mapsto \left\{ \begin{array}{ll} j ,& \text{if } j < i, \\ j+1 ,& \text{if } j \geq i. \end{array} \right. , \qquad 0 \leq i \leq n. \]
Informally, $d_n^i$ skips $i$. When it is clear from context, we sometimes omit the subscript $n$ and just write $d^i$. We call the $d^i$ \emph{coboundary} maps. Now let us define
\[ s^i_n : [n+1] \to [n], \qquad j \mapsto \left\{ \begin{array}{ll} j ,& \text{if } j < i, \\ j-1,& \text{if } j \geq i. \end{array} \right. , \qquad 0 \leq i \leq n. \]
Informally, $s_n^i$ repeats $i$. Just like with $d^i$ we write $s^i$ if there's no ambiguity over what the value of $n$ is. We call the $s^i$ \emph{codegeneracy} maps.

\begin{proposition}[The cosimplicial identities]
\label{cosimplicial identities}
The following diagrams commute. TODO
\[ \begin{tikzcd}
\left[n-1\right] \arrow[r, "d^i_n"] \arrow[d, swap, "d^{j+1}_n"] & \left[n\right] \arrow[d, "d_{n+1}^j"] \\
\left[n\right] \arrow[r, swap, "d^i_{n+1}"] & \left[n+1\right]
\end{tikzcd} \]
\end{proposition}
\begin{proof}
Easy diagram chasing.
\end{proof}

\begin{definition}
We call the presheaves of $\Delta$, i.e. the objects of $\widehat{\Delta}$ \emph{simplicial sets}.
\end{definition}

Given a finite $T_0$ space $X$, the set $X$ is partially ordered. Thus we may define a functor $\mathcal{K}$ given by
\[ \mathcal{K} : \mathbf{FinTop} \to \widehat{\Delta}, \qquad X \mapsto \Hom_{\mathbf{Poset}}\left(\cdot, X\right). \]
More explicitly, the contravariant functor $\mathcal{K}X$ is defined, for each $n \in \N$, by
\[ \left( \mathcal{K}X \right)_n = \Hom_{\mathbf{Poset}}\left([n], X \right), \]
and for each monotonic map $f : [m] \to [n]$,
\[ (\mathcal{K}X)(f) : \Hom_{\mathbf{Poset}}\left([n], X \right) \to \Hom_{\mathbf{Poset}} \left( [m], X \right), \qquad \alpha \mapsto \alpha \circ f. \]
We also have a functor
\[ \Sing : \mathbf{Top} \to \widehat{\Delta}, \qquad X \mapsto \left( [n] \mapsto \Hom_{\mathbf{Top}}\left(\Delta_n, X\right) \right). \]
